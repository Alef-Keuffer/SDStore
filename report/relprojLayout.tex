
\documentclass[11pt,a4paper]{report}%especifica o tipo de documento que tenciona escrever: carta, artigo, relatório... neste caso é um relatório
% [11pt,a4paper] Define o tamanho principal das letras do documento. caso não especifique uma delas, é assumido 10pt
% a4paper -- Define o tamanho do papel.

\usepackage[portuges]{babel}%Babel -- irá activar automaticamente as regras apropriadas de hifenização para a língua todo o
                                   %-- o texto gerado é automaticamente traduzido para Português.
                                   %  Por exemplo, “chapter” irá passar a “capítulo”, “table of contents” a “conteúdo”.
                                   % portuges -- específica para o Português.
\usepackage[utf8]{inputenc} % define o encoding usado texto fonte (input)--usual "utf8" ou "latin1

\usepackage{graphicx} %permite incluir graficos, tabelas, figuras
\usepackage{url} % para utilizar o comando \url{}
\usepackage{enumerate} %permite escolher, nas listas enumeradas, se os iems sao marcados com letras ou numeros-romanos em vez de numeracao normal

%\usepackage{apalike} % gerar biliografia no estilo 'named' (apalike)

\usepackage{color} % Para escrever em cores

\usepackage{multirow} %tabelas com multilinhas
\usepackage{array} %formatação especial de tabelas em array

\usepackage[pdftex]{hyperref} % transformar as referências internas do seu documento em hiper-ligações.

%Exemplos de fontes -- nao e vulgar mudar o tipo de fonte
%\usepackage{tgbonum} % Fonte de letra: TEX Gyre Bonum
%\usepackage{lmodern} % Fonte de letra: Latin Modern Sans Serif
%\usepackage{helvet}  % Fonte de letra: Helvetica
%\usepackage{charter} % Fonte de letra:Charter

\definecolor{saddlebrown}{rgb}{0.55, 0.27, 0.07} % para definir uma nova cor, neste caso 'saddlebrown'

\usepackage{listings}  % para utilizar blocos de texto verbatim no estilo 'listings'
%paramerização mais vulgar dos blocos LISTING - GENERAL
\lstset{
	basicstyle=\small, %o tamanho das fontes que são usadas para o código
	numbers=left, % onde colocar a numeração da linha
	numberstyle=\tiny, %o tamanho das fontes que são usadas para a numeração da linha
	numbersep=5pt, %distancia entre a numeração da linha e o codigo
	breaklines=true, %define quebra automática de linha
    frame=tB,  % caixa a volta do codigo
	mathescape=true, %habilita o modo matemático
	escapeinside={(*@}{@*)} % se escrever isto  aceita tudo o que esta dentro das marcas e nao altera
}
%
%\lstset{ %
%	language=Java,							% choose the language of the code
%	basicstyle=\ttfamily\footnotesize,		% the size of the fonts that are used for the code
%	keywordstyle=\bfseries,					% set the keyword style
%	%numbers=left,							% where to put the line-numbers
%	numberstyle=\scriptsize,				% the size of the fonts that are used for the line-numbers
%	stepnumber=2,							% the step between two line-numbers. If it's 1 each line
%											% will be numbered
%	numbersep=5pt,							% how far the line-numbers are from the code
%	backgroundcolor=\color{white},			% choose the background color. You must add \usepackage{color}
%	showspaces=false,						% show spaces adding particular underscores
%	showstringspaces=false,					% underline spaces within strings
%	showtabs=false,							% show tabs within strings adding particular underscores
%	frame=none,								% adds a frame around the code
%	%abovecaptionskip=-.8em,
%	%belowcaptionskip=.7em,
%	tabsize=2,								% sets default tabsize to 2 spaces
%	captionpos=b,							% sets the caption-position to bottom
%	breaklines=true,						% sets automatic line breaking
%	breakatwhitespace=false,				% sets if automatic breaks should only happen at whitespace
%	title=\lstname,							% show the filename of files included with \lstinputlisting;
%											% also try caption instead of title
%	escapeinside={\%*}{*)},					% if you want to add a comment within your code
%	morekeywords={*,...}					% if you want to add more keywords to the set
%}

\usepackage{xspace} % deteta se a seguir a palavra tem uma palavra ou um sinal de pontuaçao se tiver uma palavra da espaço, se for um sinal de pontuaçao nao da espaço

\parindent=0pt %espaço a deixar para fazer a  indentação da primeira linha após um parágrafo
\parskip=2pt % espaço entre o parágrafo e o texto anterior

\setlength{\oddsidemargin}{-1cm} %espaço entre o texto e a margem
\setlength{\textwidth}{18cm} %Comprimento do texto na pagina
\setlength{\headsep}{-1cm} %espaço entre o texto e o cabeçalho
\setlength{\textheight}{23cm} %altura do texto na pagina

% comando '\def' usado para definir abreviatura (macros)
% o primeiro argumento é o nome do novo comando e o segundo entre chavetas é o texto original, ou sequência de controle, para que expande
\def\darius{\textsf{Darius}\xspace}
\def\antlr{\texttt{AnTLR}\xspace}
\def\so{\emph{Sistemas Operativos}\xspace}
\def\titulo#1{\section{#1}}    %no corpo do documento usa-se na forma '\titulo{MEU TITULO}'
\def\area#1{{\em \'{A}rea: #1}\\[0.2cm]}
\def\resumo{\underline{Resumo}:\\ }

%\input{LPgeneralDefintions} %permite ler de um ficheiro de texto externo mais definições

\title{Sistemas Operativos\\
      2º Licenciatura em Ciências da Computação \\
      \textbf{Grupo 20 - Trabalho Prático}\\ SDStore: Relatório de Desenvolvimento
      } %Titulo do documento
%\title{Um Exemplo de Artigo em \LaTeX}
\author{Alef Keuffer\\ (A91683) \and Alexandre Baldé\\ (A70373)
         \and Ivo Lima\\ (A90214)
       } %autores do documento
\date{\today} %data

\begin{document} % corpo do documento
\maketitle % apresentar titulo, autor e data

\begin{abstract}  % resumo do documento
Neste relatório explicar-se-á a abordagem utilizada para construir o serviço de
armazenamento seguro de ficheiros SDStore, utilizando a linguagem C e os conhecimentos
práticos desenvolvidos nas UC de \so.
\end{abstract}

\tableofcontents % Insere a tabela de indice
%\listoffigures % Insere a tabela de indice figuras
%\listoftables % Insere a tabela de indice tabelas

\chapter{Introdução} \label{chap:intro} %referência cruzada

A estrutura do relatório será a seguinte:

\begin{description}  % Item com descrição
  \item[Problema] o problema que se quer resolver e o objetivo do projeto
  \item[Descrição] da solução implementada, e das decisões tomadas
  \item[] 
  \item[Estrutura do documento] o que é abordado em cada capitulo.
\end{description}


% colocar omitido um url de um site ou de um documento
Para mais informações sobre LATEX consultar o
 \href{http://www.ptep-online.com/ctan/lshort_port.pdf}{livro}.\\

%Colocar url de um site
 Para mais informações sobre LATEX
 consultar o livro\footnote{acessível em \url{http://www.ptep-online.com/ctan/lshort_port.pdf}}.


\section*{Estrutura do Relatório}
explicar como está organizado o documento, referindo os capítulos existentes em~\cite{deransart:1990}
e a sua articulação explicando o conteúdo de cada um.
No capítulo~\ref{chap:analiseEspecificacao} faz-se uma análise detalhada do problema proposto
de modo a poder-se especificar  as entradas, resultados e formas de transformação.\\
etc. \ldots\\ % reticencias
No capítulo~\ref{concl} termina-se o relatório com uma síntese do que foi dito,
as conclusões e o trabalho futuro



\chapter{Análise e Especificação} \label{chap:analiseEspecificacao} %capitulo e referencia cruzada
\section{Descrição informal do problema} \label{sec:descricaoProblema} %seccao e referencia cruzada
\section{Especificação do Requisitos}
\subsection{Dados} \label{subsec:dados} %subseccao e referencia cruzada
\subsection{Pedidos}
\subsection{Relações}

\chapter{Concepção/desenho da Resolução}
\section{Estruturas de Dados}
\section{Algoritmos}

\chapter{Codificação e Testes}
\section{Alternativas, Decisões e Problemas de Implementação}
\section{Testes realizados e Resultados}
Mostram-se a seguir alguns testes feitos (valores introduzidos) e
os respectivos resultados obtidos:

%\VerbatimInput{teste1.txt}

\chapter{Conclusão} \label{concl}
Síntese do Documento~\cite{araujo:2018,martini:2018}.\\
Estado final do projecto; Análise crítica dos resultados~\cite{Sto77a}.\\
Trabalho futuro.

\appendix % apendice
\chapter{Código do Programa}

Lista-se a seguir o CÓDIGO \antlr~\cite{antlr:2016} do programa
\darius~\cite{maskin:1985} que foi desenvolvido.
\begin{verbatim}
public class Aula()
  {
    int n, m;
    int max(int a, int b)
      {
       ......
       return(max);
      }
  }
\end{verbatim}

Lista-se a seguir UM TEXTO (COM O COMANDO VERBATIN)
\begin{verbatim}
      aqui deve aparecer o código do programa,
      tal como está formato no ficheiro-fonte "darius.java"
      um pouco de matematica $\$$
      caso indesejável $\varepsilon$
\end{verbatim}

\begin{lstlisting}[caption={Exemplo de uma Listagem}, label={lstExe1}]
      ou entao aparecer aqui neste sitio um pouco de matematica $\$$
      como alternativa ao anterior.
      e aqui mais um teste $\varepsilon$
\end{lstlisting}

\newpage

%-- Fim do documento -- inserção das referencias bibliográficas

%\bibliographystyle{plain} % [1] Numérico pela ordem de citação ou ordem alfabetica
\bibliographystyle{alpha} % [Hen18] abreviação do apelido e data da publicação
%\bibliographystyle{apalike} % (Araujo, 2018) apelido e data da publicação
                             % --para usar este estilo descomente no inicio o comando \usepackage{apalike}

\bibliography{bibLayout} %input do ficheiro de referencias bibliograficas

\end{document} 